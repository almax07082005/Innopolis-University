\documentclass{article}

% Language setting
% Replace `english' with e.g. `spanish' to change the document language
\usepackage[english]{babel}

% Set page size and margins
% Replace `letterpaper' with `a4paper' for UK/EU standard size
\usepackage[letterpaper,top=2cm,bottom=2cm,left=3cm,right=3cm,marginparwidth=1.75cm]{geometry}

% Useful packages
\usepackage{amsmath}
\usepackage{graphicx}
\usepackage[colorlinks=true, allcolors=blue]{hyperref}
\usepackage{amstext} % for \text macro
\usepackage{array}   % for \newcolumntype macro
\newcolumntype{L}{>{$}l<{$}} % math-mode version of "l" column type
\usepackage{makecell}
\usepackage{caption}

\usepackage[outline]{contour}
\contourlength{0.25pt}

\usepackage{listings} %For code in appendix
\lstset
{ %Formatting for code in appendix
    language=C++,
    basicstyle=\footnotesize,
    numbers=left,
    stepnumber=1,
    showstringspaces=false,
    tabsize=2,
    breaklines=true,
    breakatwhitespace=false,
}

\usepackage{mathtools}
\DeclarePairedDelimiter\ceil{\lceil}{\rceil}
\DeclarePairedDelimiter\floor{\lfloor}{\rfloor}

\title{DSA. Problem solutions. Week 4.}
\author{Maksim Al Dandan}

\begin{document}
\maketitle


% TASK 1
\section{Task 1}
\subsection{Statement}
Compute asymptotic worst case time complexity of the solve procedure:

\begin{enumerate}
    \setlength\itemsep{0em}
    \item[(i)] Express the running time of the solve procedure $T(n)$ as a recurrence relation.
    \item[(ii)] Find the asymptotic complexity of $T(n)$ using the master theorem. 
    \item[(iii)] Specify which case of the master theorem applies (if any).
    \item[(iv)] Write down conditions for this case that need to be checked (write down specialized version for a particular recurrence).
    \item[(v)] Prove that the conditions are satisfied. For asymptotic notation, each property used in a proof must be either proven explicitly or properly referenced (e.g. by citing [Cormen] or a particular Lecture/Tutorial slide).
\end{enumerate}

\begin{lstlisting}[escapeinside={(*}{*)}]
    /* A is a 0-indexed array
     * n is the number of elements in A */
    solve(A, n):
      helper(A, 0 , n - 1)
 
    helper(A, l, r):
      if l > r:
        return 
      else:
        k := (*$\ceil{(r - l + 1) / 4}$*) 
        for j from 0 to 3 /* inclusive */
          helper(A, l + j * k, min(r, l + (j + 1) * k))
        for a from l to r
          m := i
          for b from l to r
            if A[b] (*$\geq$*) A[m]:
              m = b
          exchange A[m] with A[a]
\end{lstlisting}

\subsection{Solution}

\hspace{0.5cm} Running time of the "solve" procedure: \(T(n) = 4T(\frac{n}{4}) + n^2\) \\
We divide array by four parts and call the helper function itself only once within the for loop for exactly 4 times. That's way \(a = b = 4\). \\

$f(n) = n^2$ is because the code from the 13th line till the end takes asymptotic worst case time complexity $O(n^2)$. \\

Then let's try to apply the Master theorem. \\ \\
$a = 4, b = 4, f(n) = n^2, n^{\log_{b}a} = n^{\log_{4}4} = n$ \\
\textbf{[Case 1] $\implies$} $f(n) = n^2 = O(n^{1 - \epsilon})$, $\epsilon > 0$ : $\epsilon$ doesn't exist \\
\textbf{[Case 2] $\implies$} $f(n) = n^2 = \Theta(n\times \log^k{n})$, $k \geq 0$ : $k$ doesn't exist \\
\textbf{[Case 3] $\implies$} $f(n) = n^2 = \Omega(n^{1 + \epsilon})$, $\epsilon > 0$ : $\epsilon = 0.5$ (exist), then we nee to check \textit{regularity condition}. \\
\begin{equation}
    4\times (\frac{n}{4})^2 \leq cn^2
\end{equation}
\begin{equation}
    c \geq \frac{1}{4}, c < 1
\end{equation}
For instance, $c = 0.5$. Hence, \textit{regularity condition} holds and, thus, we can compute $T(n)$. \\
$T(n) = \Theta(f(n)) = \Theta(n^2)$. \\

\textbf{Answer: $T(n) = \Theta(n^2)$.}

% TASK 2
\section{Task 2}
\subsection{Statement}
For each of the following recurrences, apply the master theorem [Cormen, Theorem 4.1] yielding a closed form formula for the asymptotic complexity of $T(n)$. Assume that $T(n) = 1$ when $n < 10$ for all recurrences below. You must specify the following:
\begin{itemize}
    \setlength\itemsep{0em}
    \item Can the theorem be applied to the recurrence?
    \item If yes, then
    \begin{enumerate}
        \setlength\itemsep{0em}
        \item [(i)] Which case of the master theorem applies?
        \item [(ii)] Which conditions for this case that need to be checked (write down specialized version for a particular recurrence).
        \item [(iii)] Prove that the condition is satisfied. For asymptotic notation, each property used in a proof \contour{black}{must} be either proven explicitly or properly referenced (e.g. by citing [Cormen] or a particular Lecture/Tutorial slide).
        \item [(iv)] Provide asymptotic complexity for $T(n)$ using $\Theta$-notation.
    \end{enumerate}
    \item Otherwise, provide an explicit justification, explaining why the theorem cannot be applied.
\end{itemize}
\begin{itemize}
    \setlength\itemsep{0em}
    \item [(a)] $T(n) = 2T(n/3) + \log_2n$
    \item [(b)] $T(n) = 7T(n/49) + \sqrt{n}\cdot\log_2^2 n$
    \item [(c)] $T(n) = 4T(n/3) + n^2$
    \item [(d)] $T(n) = 2T(\sqrt{n}) + n$
    \item [(e)] $T(n) = \frac{1}{2}T(2n) + n\log_2 n$
\end{itemize}

\subsection{Solution}
\subsubsection{Point (a)}

$a = 2, b = 3, f(n) = \log_{2}n, n^{\log_{b}a} = n^{\log_{3}2} = n \approx n^{0.631}$ \\
\textbf{[Case 1] $\implies$} $f(n) = \log_{2}n = O(n^{0.631 - \epsilon})$, $\epsilon > 0$ : $\epsilon = 0.1$ (exist), then we can compute $T(n)$. \\ \\
$T(n) = \Theta(n^{\log_32})$

\subsubsection{Point (b)}

$a = 7, b = 49, f(n) = \sqrt{n} \times \log^2_2n, n^{\log_{b}a} = n^{\log_{49}7} = \sqrt{n}$ \\
\textbf{[Case 1] $\implies$} $f(n) = \sqrt{n} \times \log^2_2n = O(n^{\frac{1}{2} - \epsilon})$, $\epsilon > 0$ : $\epsilon$ doesn't exist \\
\textbf{[Case 2] $\implies$} $f(n) = \sqrt{n} \times \log^2_2n = \Theta(\sqrt{n} \times \log^k{n})$, $k \geq 0$ : $k = 2$ (exist), then we can compute $T(n)$. \\ \\
$T(n) = \Theta(\sqrt{n} \times \log^{3}_{2}n)$

\subsubsection{Point (c)}

$a = 4, b = 3, f(n) = n^2, n^{\log_{b}a} = n^{\log_{3}4} \approx n^{1.262}$ \\
\textbf{[Case 1] $\implies$} $f(n) = n^2 = O(n^{1.262 - \epsilon})$, $\epsilon > 0$ : $\epsilon$ doesn't exist \\
\textbf{[Case 2] $\implies$} $f(n) = n^2 = \Theta(n\times \log^k{n})$, $k \geq 0$ : $k$ doesn't exist \\
\textbf{[Case 3] $\implies$} $f(n) = n^2 = \Omega(n^{1.262 + \epsilon})$, $\epsilon > 0$ : $\epsilon = 0.1$ (exist), then we nee to check \textit{regularity condition}. \\
\setcounter{equation}{0}
\begin{equation}
    4\times (\frac{n}{3})^2 \leq cn^2
\end{equation}
\begin{equation}
    c \geq \frac{4}{9}, c < 1
\end{equation}
For instance, $c = \frac{5}{9}$. Hence, \textit{regularity condition} holds and, thus, we can compute $T(n)$. \\ \\
$T(n) = \Theta(f(n)) = \Theta(n^2)$.

\subsubsection{Point (d)}

$a = 2, b = 1, f(n) = n$ \\
Hence, cannot be computed because by the formula of Master theorem $b$ should be constant and $b > 1$.

\subsubsection{Point (e)}

$a = \frac{1}{2}, b = \frac{1}{2}, f(n) = n\log_2n$ \\
Hence, cannot be computed because by the formula of Master theorem $b > 1$.

\subsection{Answer}
\begin{itemize}
    \setlength\itemsep{0em}
    \item [(a)] $T(n) = \Theta(n^{\log_32})$
    \item [(b)] $T(n) = \Theta(\sqrt{n} \times \log^{3}_{2}n)$
    \item [(c)] $T(n) = \Theta(n^2)$
    \item [(d)] No answer
    \item [(e)] No answer
\end{itemize}

\section*{References}
[Cormen] T. H. Cormen, C. E. Leiserson, R. L. Rivest and C. Stein. Introduction to Algorithms, Fourth Edition. The MIT Press 2022

\end{document}