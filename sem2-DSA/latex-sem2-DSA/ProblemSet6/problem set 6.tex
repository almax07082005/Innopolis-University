\documentclass[10pt]{article}
\usepackage[utf8]{inputenc}
\usepackage[T1]{fontenc}
\usepackage{amsmath}
\usepackage{amsfonts}
\usepackage{amssymb}
\usepackage[version=4]{mhchem}
\usepackage{stmaryrd}

\title{Data Structures and Algorithms Spring 2024 — Problem Sets }


\author{by Nikolai Kudasov}



%New command to display footnote whose markers will always be hidden
\let\svthefootnote\thefootnote
\newcommand\blfootnotetext[1]{%
  \let\thefootnote\relax\footnote{#1}%
  \addtocounter{footnote}{-1}%
  \let\thefootnote\svthefootnote%
}

%Overriding the \footnotetext command to hide the marker if its value is `0`
\let\svfootnotetext\footnotetext
\renewcommand\footnotetext[2][?]{%
  \if\relax#1\relax%
    \ifnum\value{footnote}=0\blfootnotetext{#2}\else\svfootnotetext{#2}\fi%
  \else%
    \if?#1\ifnum\value{footnote}=0\blfootnotetext{#2}\else\svfootnotetext{#2}\fi%
    \else\svfootnotetext[#1]{#2}\fi%
  \fi
}

\begin{document}
\maketitle


\section*{Week 6. Problem set}
\begin{enumerate}
  \item Consider a modification of the randomized QUICK-SoRT algorithm [Cormen, §7.3] that stops recursion when the size of subarray becomes less than or equal to $k(k \leq n)$. For arrays of size $\leq k$, the modified algorithm performs BUBBLE-SORT. Answer the following questions about the modified algorithm:
\end{enumerate}

(a) What is the worst case time complexity in terms of $n$ and $k$ ?

(b) What is the best case time complexity in terms of $n$ and $k$ ?

(c) What is the average ${ }^{1}$ case time complexity in terms of $n$ and $k$ ?

The answer should be given using $\Theta$-notation.

Provide a brief justification for each case (1-2 sentences).

\begin{enumerate}
  \setcounter{enumi}{1}
  \item Apply Counting-Sort to the following input array where each column corresponds to one item with its numeric key and single-character satellite data:
\end{enumerate}

\begin{center}
\begin{tabular}{|c|c|c|c|c|c|c|c|c|c|c|c|}
\hline
2 & 8 & 2 & 6 & 7 & 6 & 3 & 4 & 1 & 6 & 2 & 0 \\
\hline
$\mathrm{D}$ & $\mathrm{T}$ & $\mathrm{O}$ & $\mathrm{G}$ & $\mathrm{A}$ & $\mathrm{R}$ & $\mathrm{N}$ & $\mathrm{G}$ & $\mathrm{R}$ & $\mathrm{E}$ & $\mathrm{I}$ & $\mathrm{U}$ \\
\hline
\end{tabular}
\end{center}

You must demonstrate the final state of the auxiliary arrays used in the algorithm, as well as the output of the array.

\section*{References}
[Cormen] T. H. Cormen, C. E. Leiserson, R. L. Rivest and C. Stein. Introduction to Algorithms, Fourth Edition. The MIT Press 2022
\footnotetext{${ }^{1}$ assuming all elements in the input array are distinct and any initial order in the array is equally likely
}


\end{document}