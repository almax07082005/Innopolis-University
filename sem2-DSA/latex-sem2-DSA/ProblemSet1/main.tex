\documentclass{article}

% Language setting
% Replace `english' with e.g. `spanish' to change the document language
\usepackage[english]{babel}

% Set page size and margins
% Replace `letterpaper' with `a4paper' for UK/EU standard size
\usepackage[letterpaper,top=2cm,bottom=2cm,left=3cm,right=3cm,marginparwidth=1.75cm]{geometry}

% Useful packages
\usepackage{amsmath}
\usepackage{graphicx}
\usepackage[colorlinks=true, allcolors=blue]{hyperref}
\usepackage{amstext} % for \text macro
\usepackage{array}   % for \newcolumntype macro
\newcolumntype{L}{>{$}l<{$}} % math-mode version of "l" column type
\usepackage{makecell}
\usepackage{caption}

\usepackage{listings} %For code in appendix
\lstset
{ %Formatting for code in appendix
    language=Scilab,
    basicstyle=\footnotesize,
    numbers=left,
    stepnumber=1,
    showstringspaces=false,
    tabsize=2,
    breaklines=true,
    breakatwhitespace=false,
}

\title{DSA. Problem solutions. Week 1.}
\author{Maksim Al Dandan}

\begin{document}
\maketitle

\section{Task 1}

\subsection{Statement}

Compute asymptotic worst case time complexity of the following algorithm (see pseudocode conventions in [\textcolor{blue}{Cormen}, Section 2.1]). You \textbf{must} use \(\Theta\)-notation. For justification, provide execution cost and frequency count for each line in the body of the \textcolor{blue}{secret} procedure. Optionally, you may provide the details for the computation of the running time \(T(n)\) for worst case scenario. Proof for the asymptotic bound is not required for this exercise.

\begin{lstlisting}[mathescape=true]
      /* A is a 0-indexed array,
       * n is the number of items in A */
      secret(A, n):
        k := 0
        for i = 1 to n-1
          k := k + 1
          j := i
          while j < n and A[j-1] $\ge$ A[j]
            j := 2 * j
          exchange A[i] with A[min(j, n - 1)]
\end{lstlisting}

\subsection{Solution}
\begin{table}[h!]
    \centering
    \begin{tabular}{| L | L |}
    \hline
    \text{Cost} & \text{Times} \\
    \hline
    c_4 & 1 \\
    c_5 &  n \\
    c_6 &  n - 1 \\
    c_7 &  n - 1 \\
    c_8 &  n(n - 1) \\
    c_9 &  (n - 1)^2 \\
    c_{10} &  n - 1 \\
    \hline
    \end{tabular}
\end{table}

\begin{equation}
    T(n) = c_4 * 1 + c_5 * n + c_6 * (n - 1) + c_7 * (n - 1) + c_8 * n(n - 1) + c_9 * (n - 1)^2 + c_{10} * (n - 1)
\end{equation}
\begin{equation}
    T(n) = c_4 + nc_5 + c_6(n - 1) + c_7(n - 1) + nc_8(n - 1) + c_9(n - 1)^2 + c_{10}(n - 1)
\end{equation} \\
\textbf{Answer: $T(n) = \Theta(n^2)$}

\section{Task 2}

\subsection{Statement}

Indicate, for each pair of expressions \((A, B)\) in the table below whether \(A = O(B)\),\( A = o(B)\), \(A = \Omega(B)\), \(A = \omega(B)\), or \(A = \Theta(B)\). Write your answer in the form of the table with \textit{yes} or \textit{no} written in each box.

\subsection{Answer}

\begin{table}[h!]
    \centering
    \begin{tabular}{|c|c||c|c|c|c|c|} \hline
        \(A\) & \(B\) & \(A = O(B)\) & \( A = o(B)\) & \(A = \Omega(B)\) & \(A = \omega(B)\)& \(A = \Theta(B)\) \\ \hline \hline
        \(1.0001^n\) & \(n^{1000}\) & yes & yes & no & no & no \\ \hline 
        \(3\) & \((1+1/n)^n\) & yes & yes & no & no & no \\ \hline 
        \(n^{\sin n}\) & \(\log_2n\) & no & no & no & no & no \\ \hline 
        \(\log_2^3n\) & \(\sqrt[6]n\) & no & no & yes & yes & no \\ \hline
    \end{tabular}
\end{table}

\section{Task 3}
\subsection{Statement}

Let \(f\) and \(g\) be functions from positive integers to positive reals. Assume \(g(n) > n\) for \(n > 0\). Using definition of asymptotic notation, prove formally that

\[max(f(n) + \sqrt{n}, g(n) - n) = O(f(n) + g(n))\]

\subsection{Solution}
\textbf{Proof:} \\

We need to show that there exist constants c and $n_0$, such that for all n $\geq n_0$ we have. \\
Let us consider two cases: \\ \\
1. $max = f(n) + \sqrt{n}$ \\
It implies that: $f(n) + \sqrt{n} \leq f(n) + g(n)$, where g(n) $>$ n \\
Thus, $f(n) + \sqrt{n} \leq f(n) + n$ (at least) \\
But n grows faster than $\sqrt{n}$, hence it works. \\ \\
2. $max = g(n) - n$ \\
It implies that: $g(n) - n \leq f(n) + g(n)$, where g(n) $>$ n \\
Thus, $f(n) \geq -n$, but f(n) due to condition contains only of positive reals \\
Hence, it works always. \\ \\
\textbf{QED}

\end{document}