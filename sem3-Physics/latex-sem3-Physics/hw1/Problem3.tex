\section{Problem 3}

\vspace{7mm}

To determine the speed of the river relative to the ground, we need to use concepts of relative motion. The boat travels upstream and drops a bottle, which then moves downstream with the river current. The fisherman, upon returning downstream, finds the bottle 5 km from the bridge. We will set up equations based on the relative speeds of the boat and the river, and solve for the river's speed. \\

We denote \( v_r \) as the speed of the river relative to the ground and \( v_b \) as the speed of the boat relative to the ground. The distance traveled by the bottle downstream in half an hour is \( 0.5 \times v_r \) km. The fisherman goes up the river for the following distance:

\[
S_{up} = (v_b - v_r) \times 0.5
\]

The time \( t_{toBridge} \) for the fisherman to reach the bridge again is:

\[
t_{toBridge} = \frac{(v_b - v_r) \times 0.5}{v_b + v_r}
\]

The time \( t_{toBottle} \) for the fisherman to reach the bottle from the bridge is:

\[
t_{toBottle} = \frac{5}{v_b + v_r}
\]

The overall time \( t_{toBridge} + t_{toBottle} + t_{fromBridge} \) is:

\[
t = \frac{(v_b - v_r) \times 0.5 + 5}{v_b + v_r} + 0.5 = \frac{5}{v_r}
\]

By solving that, everything except $v_r$ will be canceled out, so the answer is:

\[
v_r = \boxed{5 \frac{km}{h}}
\]
