\section{Problem 2}

\textbf{Solution:}

\textbf{2.1. Position of maximum tension}

In vertical circular motion, the tension in the string is due to both the centripetal force and the gravitational force. The forces acting on the object are:

- \textbf{Centripetal force}: which is directed towards the center of the circle.
- \textbf{Gravitational force}: always acting downward.

At the \textbf{top} of the circular trajectory, both the gravitational force and the centripetal force act in the same direction, towards the center. At the \textbf{bottom}, they act in opposite directions. Therefore, the \textbf{maximum tension} occurs at the \textbf{bottom of the circle}, where the gravitational force opposes the centripetal force.

\[
\text{Thus, the position of maximum tension is at the bottom of the circle.}
\]

\textbf{2.2. Maximum speed before the string breaks}

At the bottom of the circular path, the tension in the string is given by:

\[
T = \frac{mv^2}{R} + mg
\]
where:
- \( T \) is the tension in the string,
- \( m = 0.25 \, \text{kg} \) is the mass of the object,
- \( v \) is the speed of the object,
- \( R = 0.7 \, \text{m} \) is the radius of the circle,
- \( g = 9.8 \, \text{m/s}^2 \) is the acceleration due to gravity.

To find the maximum speed \( v_{\text{max}} \) when the string does not break, we rearrange the equation:

\[
T_{\text{max}} = \frac{mv_{\text{max}}^2}{R} + mg
\]

Solving for \( v_{\text{max}} \):

\[
v_{\text{max}} = \sqrt{\frac{(T_{\text{max}} - mg)R}{m}}
\]

Substituting the known values:

\[
v_{\text{max}} = \sqrt{\frac{(30 - 0.25 \times 9.8) \times 0.7}{0.25}}
\]

\[
v_{\text{max}} = \sqrt{\frac{(30 - 2.45) \times 0.7}{0.25}} = \sqrt{\frac{27.55 \times 0.7}{0.25}} = \sqrt{\frac{19.285}{0.25}} = \sqrt{77.14}
\]

\[
v_{\text{max}} \approx 8.78 \, \text{m/s}
\]

Thus, the maximum speed is \( \boxed{8.8 \, \text{m/s}} \).