\section{Problem 3}

\textbf{Step 1: Applying Newton's second law}

The equation of motion for the boat is:

\[
m \frac{dv}{dt} = F_{\text{engine}} - kv
\]

where:
\begin{itemize}
    \item $m = 3000 \, \text{kg}$ (mass of the boat),
    \item $k = 100 \, \text{kg/s}$ (drag coefficient),
    \item $F_{\text{engine}}$ is the constant engine force,
    \item $v(t)$ is the velocity as a function of time.
\end{itemize}

Rearranging the equation:

\[
\frac{dv}{dt} + \frac{k}{m}v = \frac{F_{\text{engine}}}{m}
\]

This is a first-order linear differential equation. \\

\textbf{Step 2: Solving the differential equation}

The solution to this equation is of the form:

\[
v(t) = V_{\text{max}} \left(1 - e^{-\frac{k}{m} t}\right)
\]

where $V_{\text{max}} = \frac{F_{\text{engine}}}{k}$ is the maximum velocity the boat can reach when the drag force equals the engine force.

\textbf{Step 3: Determining the engine force}

We know that after $t = 33$ seconds, the velocity is $v(33) = 2 \, \text{m/s}$. Substituting this into the velocity equation:

\[
2 = V_{\text{max}} \left(1 - e^{-\frac{100}{3000} \times 33}\right)
\]

Simplifying the exponent:

\[
2 = V_{\text{max}} \left(1 - e^{-1.1}\right)
\]

Using $e^{-1.1} \approx 0.3329$:

\[
2 = V_{\text{max}} \times (1 - 0.3329)
\]

\[
2 = V_{\text{max}} \times 0.6671
\]

Thus,

\[
V_{\text{max}} = \frac{2}{0.6671} \approx \boxed{2.999 \, \text{m/s}}
\]
